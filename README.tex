%%%%%%%%%%%%%%%%%%%%%%%%%%%%%%%%%%%%%%%%%
% NIH Grant Proposal for the Specific Aims and Research Plan Sections
% LaTeX Template
% Version 1.0 (21/10/13)
%
% This template has been downloaded from:
% http://www.LaTeXTemplates.com
%
% Original author:
% Erick Tatro (erickttr@gmail.com) with modifications by:
% Vel (vel@latextemplates.com)
%
% Adapted from:
% J. Hrabe (http://www.magalien.com/public/nih_grants_in_latex.html)
%
% License:
% CC BY-NC-SA 3.0 (http://creativecommons.org/licenses/by-nc-sa/3.0/)
%
%%%%%%%%%%%%%%%%%%%%%%%%%%%%%%%%%%%%%%%%%

%----------------------------------------------------------------------------------------
%	PACKAGES AND OTHER DOCUMENT CONFIGURATIONS
%----------------------------------------------------------------------------------------

\documentclass[11pt,notitlepage]{article}

% A note on fonts: As of 2013, NIH allows Georgia, Arial, Helvetica, and Palatino Linotype. LaTeX doesn't have Georgia or Arial built in; you can try to come up with your own solution if you wish to use those fonts. Here, Palatino & Helvetica are available, leave the font you want to use uncommented while commenting out the other one.
\usepackage{palatino} % Palatino font
%\usepackage{helvet} % Helvetica font
\renewcommand*\familydefault{\sfdefault} % Use the sans serif version of the font
\usepackage[T1]{fontenc}
\linespread{1.05} % A little extra line spread is better for the Palatino font

%\usepackage{lipsum} % Used for inserting dummy 'Lorem ipsum' text into the template
\usepackage{amsfonts, amsmath, amsthm, amssymb} % For math fonts, symbols and environments
\usepackage{graphicx} % Required for including images
\usepackage{booktabs} % Top and bottom rules for table
\usepackage{wrapfig} % Allows in-line images
\usepackage[labelfont=bf]{caption} % Make figure numbering in captions bold
\usepackage{xspace}
\usepackage[top=0.6in,bottom=0.6in,left=0.6in,right=0.6in]{geometry} % Reduce the size of the margin
\pagestyle{empty} % Remove page numbers
\usepackage[font=scriptsize]{caption}
\hyphenation{ionto-pho-re-tic iso-tro-pic fortran} % Specifies custom hyphenation points for words or words that shouldn't be hyphenated at all
\usepackage{fancyhdr}
\fancyhf{}
\pagestyle{fancy}
\rfoot{\thepage}
\renewcommand{\headrulewidth}{0pt}
\newcommand{\aseq}{{\bf a}\xspace}
\newcommand{\aseqk}{{\bf a_k}\xspace}
\usepackage[normalem]{ulem}
\usepackage[document]{ragged2e}
 
\begin{document}
%----------------------------------------------------------------------------------------
%	SPECIFIC AIMS
%----------------------------------------------------------------------------------------

To support my PhD comprehensive exam and allow initial investigation of robustness in biological sequences, I developed this dynamic programming algorithm to count the number of compatible k-mutants with the structure of a given sequence. This algorithm is a simpler and faster way to compute $Z^*_k$ in the uniform energy model, while the algorithm proposed in aim 1 of the proposal is suitable to be extended to the Turner energy model.\\
For a given sequence $\aseq$, folding into structure $s_0$, define $Z^*(i,j,\mu)$ to be the number of sequences $\aseq[i,j]$, with $\mu \leq k$ number of mutations compatible with $s_0[i,j]$. For a nucleotide pair $(x,y)$, let $m1(x,y)$ and $m2(x,y)$ respectively be the number of single-point and two-point mutations in $(x,y)$, that keep base pair compatibility. For instance $m1(G,U)=2$, because the only possible single-point mutations without destroying the base pair are $G\to A$ and $U\to C$. Therefore
\[ m1(x,y) = \begin{cases} 
      2 & if \quad (x,y)=(G,U) or (U,G) \\
      1 & otherwise
   \end{cases}
\]
\[ m2(x,y) = \begin{cases} 
      3 & if \quad (x,y)=(G,U) or (U,G) \\
      4 & otherwise
   \end{cases}
\]
The algorithm is as follows:\\
\underline{Base case}: For for $0\leq i \leq n$, define $Z^*(i,i,0)=1$ and 
\[ Z^*(i,i,1) = \begin{cases} 
      3 & if \quad i \quad is \quad unpaired  \\
      0 & otherwise
   \end{cases}
\]
\underline{Case 1:} $j$ is unpaired:
\begin{eqnarray}
Z^*(i,j,\mu) += 3 \cdot Z^*(i,j-1,\mu-1) + Z^*(i,j-1,\mu)
\end{eqnarray}
\underline{Case 2:} If $j$ is paired with position $r$ and $r<j$:
\begin{eqnarray}
Z^*(i,j,\mu) += Z^*(i,j-1,\mu) + Z^*(i,j-1,\mu-1) \cdot m1(\aseq_r,\aseq_j) +  Z^*(i,j-1,\mu-2) \cdot m2(\aseq_r,\aseq_j)
\end{eqnarray}
\underline{Case 3:} If $j$ is paired with position $r$ and $r>j$:
\begin{eqnarray}
Z^*(i,j,\mu) += Z^*(i,j-1,\mu)
\end{eqnarray}
For a valid secondary structure $s_0[i,j]$ with a balanced number of base pairs, $Z^*(i,j,\mu)$ is the number of $\mu$-mutants of $\aseq[i,j]$, compatible with $s_0[i,j]$. The time complexity of the above algorithm is $O(k n^2)$.

\section*{Usage}
Usage:\\ 
\qquad\qquad ./RNACompatible sequence structure k\\

Example:\\
\qquad\qquad ./RNACompatible.py  CCCCCCCAAAAAGGGGGGG '(((((((.....)))))))' 5\\

\end{document}
